\chapter{Project proposal}

\begin{center}
\Large
Computer Science Tripos -- Part II -- Project Proposal\\[4mm]
\LARGE
An implementation of Google Dataflow in the Elixir programming language\\[4mm]

\large
M.~N.~Jadczak \texttt{<mnj24>}, Robinson College

Originators: Alastair Beresford and Martin Kleppmann

14 October 2016
\end{center}

\vspace{5mm}

\textbf{Project Supervisor:} Dr A.~Beresford

\textbf{Director of Studies:} Dr A.~Beresford

\textbf{Project Overseers:} Dr S.~Holden  \& Dr S.~Teufel

% Main document

\section*{Introduction}

As various business services become ever more realtime, there is a strengthening shift to processing of unbounded data---running streams of data through pipelines of transformations rather than running batch processes on static data. Furthermore, there may be sophisticated requirements of the data pipelines, including the need to apply event-time ordering and window the data by features of the data points themselves. The Dataflow Model due to Google \cite{Akidau:2015} attempts to create a consolidated abstraction of such data processing, whilst allowing for flexible tuning of the system between correctness, latency and cost. The Dataflow layer manages the complicated semantics of the pipelines, delegating the actual batch and stream processing itself to external tools or ``runners''.

Since the publication of the paper, Google have released a commercial Cloud implementation of the model \cite{CloudDataflow} which comprises a distributed, tuned implementation of such a runner tuned specifically for use with the Dataflow SDK. They have, however, open-sourced the Dataflow layer itself as the Apache Beam project \cite{ApacheBeam} and aim to provide the SDK in several languages. The pipelined model in question is now being called the Beam model and a reference Java SDK is ready, with a Python SDK nearing completion. As well as interfacing with the Google Cloud Dataflow runner, the Beam project includes adapters for Apache Spark and Apache Flink, as well as an implementation of a runner which can run on a local machine.

This project aims to provide an implementation of this SDK in the Elixir programming language \cite{Elixir}, a relatively modern functional language which runs atop BEAM, the Erlang VM. Its robust macro system makes it an excellent tool for writing performant yet expressive DSLs, and it takes advantage of OTP (the Erlang runtime framework) to provide robustness and easy-to-write concurrency.

\section*{Starting point}

As described above, a detailed paper describing the Dataflow Model is available, as well as a reference Java implementation (as well as a half-finished Python one). I am familiar with the Elixir language, having read several books on the topic as well as implementing a few separate projects in the language over the Summer; however, none were of this sort or scale.

\section*{Resources required}

The only external resources which may be required are the datasets described in the possible extensions section; I have obtained verbal agreement of their availability, but they are non-critical to the project. During the evaluation phase I may also require paid execution time in the Google Cloud---I am fully willing and prepared to pay for any such usage myself where it cannot be obtained for free (as is often the case for educational use).

The project data will be secured using many redundant layers of backup:

\begin{itemize}
	\item The dissertation and code will be kept in Git repositories hosted (as private repositories) on GitHub;
	\item The working folder containing all files relevant to the project will be synced to my Dropbox account;
	\item The entire filesystem undergoes hourly backups to an external hard drive using Apple Time Machine;
	\item The entire filesystem is also daily backed up to an offsite online service, Backblaze.
\end{itemize}

I will perform the work on my personal machine, a 2013 15" Retina MacBook Pro (16GB RAM, 2.8 GHz Intel Core i7) running macOS Sierra. The hardware itself is insured, and in any case the work should be transferable to an MCS system.

\section*{Work to be done}

The main contributions of the Dataflow model are described in \cite{Akidau:2015} as follows:

\begin{itemize}
	\item A \textbf{windowing model} which supports unaligned event-time windows, and a simple API for their creation and use;
	\item A \textbf{triggering model} that binds the output times of results to runtime characteristics of the pipeline, with a powerful and flexible declarative API for describing desired triggering semantics;
	\item An \textbf{incremental processing model} that integrates retractions and updates into the windowing and triggering models described above; and
	\item \textbf{Scalable implementations} of the above atop the MillWheel streaming engine and the FlumeJava batch engine, with an external reimplementation for Google Cloud Dataflow, including an open-source SDK that is runtime-agnostic.
\end{itemize}

The concrete implementation also provides an expressive (for Java) API to the SDK which allows non-expert programmers to take advantage of this powerful system.

For clarification, my project aims only to implement the Dataflow Model segment of the project, which handles parsing and processing the pipelines. The batch/stream processing work itself will be done by a pre\"existing Java runner.

My project will involve:

\begin{enumerate}
	\item Defining a suitably complex vertical slice of the stack for initial implementation, which hits on all areas above without necessarily fully implementing all of their features. The intention is to capture all of the important elements of the system whilst perhaps not allowing certain options or settings. This will be optimised towards the intended test cases;
	\item Creating an expressive, intuitive DSL for programmers to interface with the engine, following Elixir conventions while mirroring the Java API in feature set and concepts;
	\item Implementing a small Java wrapper over the intended runner which will interface with the Elixir control layer using Erlang ports (communication over standard Unix FDs, using a well-defined binary format used to serialise Erlang values);
	\item Implementing both the DSL and the control layer slice, such that initial test cases can be run successfully and produce output similar or identical to the reference implementation;
	\item If time allows, extending the code to cover more of the features and functionality of the Beam SDK, as well as optimising the code to take full advantage of the robust and concurrent environment it is run in.
\end{enumerate}

\section*{Success criteria}

Evaluation will be possible against the reference implementation as well as possibly the commercial cloud one if necessary, using publicly accessible data from e.g.\ the Twitter firehose or OpenStreetMaps. Graphs will be produced of the performance characteristics of both systems under varying load size.

Success in this regard will comprise the successful execution of test cases defined during the preparation phase.

As well as comparison of raw performance against the reference implementation, a goal of this project is to provide a much nicer, more natural interface/DSL to the powerful Dataflow model. No external human testing of ``niceness'' will be performed, but success in this area will be achieved by the provision of a well-reasoned argument backed with examples of queries which are easier to write, understand, and better capture the essence of the query itself when written using the Elixir DSL compared to the Java API.


\section*{Possible extensions}

As well as evaluation against public data, I have secured an offer to test both my implementation and the reference one on data from the Isaac Physics \cite{Isaac} data (streaming question-answering data) as well as the firehose of device statistics from the Device Analyzer \cite{Device} project. This offer is from my project supervisor who is involved in both of these projects. These may provide a more real-world, focused example than using public data.

In addition to this, as mentioned in the implementation section, the project will start off with the implementation of a vertical slice of functionality, after whose completion it will be successful. If time allows however, fleshing out this slice to approach the full feature set supported by the Beam SDK will be desirable.


\section*{Timetable}

The planned starting date is 20/10/2016. My plan assumes the division of the work into ten work-packages, detailed below:

\begin{enumerate}

\item \textbf{Michaelmas weeks 2--4. By 02/11/16:} Read the source code of the reference implementation and study the paper; achieve familiarity with the concepts and patterns used. Produce notes which will be used for the Preparation section.

\item \textbf{Michaelmas weeks 5--6. By 16/11/16:} Identify the exact modules/sections of code which form the necessary vertical slice of features. Produce first iteration of the DSL definition. Continue working on Preparation notes, fleshing them out into a very early draft of the section.

\item \textbf{Michaelmas weeks 7--8. By 30/11/16:} Write test cases using the reference implementation and make sure they work. Replicate them using the DSL and ensure that the DSL can capture them elegantly and that the selected feature set can support them. Begin implementation. 

\item \textbf{Michaelmas vacation 1. By 21/12/16:} Perform initial internal implementation work, including the Java wrapper for communication with the runner. Adjust DSL to take the design of the system into account if necessary. Keep notes in prose, forming an early draft of the implementation section.
\item \textbf{Michaelmas vacation 22/12/16--02/01/17:} Take time off.

\item \textbf{Michaelmas vacation 2. By 18/01/17:} Continue work on implementation. Begin writing progress report and implementation of DSL itself.

\item \textbf{Lent weeks 1--2. By 01/02/17:} Write progress report. Finish internal implementation and implement the DSL itself.

\item \textbf{Lent weeks 3--5. By 22/02/17:} Implementation is complete, first evaluation results available, ideally using the test cases defined at the beginning of the project. Ensure these are representative test cases and start writing the draft of the Evaluation section. If necessary, iterate on code implementation if tests are unsatisfactory.

\item \textbf{Lent weeks 6--8. By 15/03/17:} Keep iterating on code, whether to bring it in line with the test spec or to extend its functionality. By the end of the LT the 
\item \textbf{Easter vacation 1. By 29/03/17:} Codebase, test cases and any results are be fully complete.
\item \textbf{Easter vacation 2. By 19/04/17:} Write main body of dissertation, drawing on previous drafts and notes. If time allows work on extensions.

\item \textbf{Easter weeks 0--2. By 10/05/17:} Submit full first draft to supervisor on 25/04/17, implement any comments and suggestions. Aim for early submission.

\item \textbf{Easter week 3. By 19/05/17:} If not yet submitted, finalise all documents and code in preparation for submission.

\end{enumerate}
