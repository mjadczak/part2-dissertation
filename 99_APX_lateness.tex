\chapter{Additional detail}\label{apx:aditional}

This appendix holds extra detail of some aspects of the system.
These are necessary for a working implementation but do not contribute greatly to an overview of the system, and so are excluded from the main text.

\section{Watermark hold internals}\label{apx:additional:watermark-holds}

A Transform Executor is a runtime instance of a Transform.

By default, each Transform Executor outputs a watermark which is equivalent to its input watermark, unless the output watermark is held back by a \emph{watermark hold}.
In that case, the output watermark is the earlier of the two.
Watermark holds provide a mechanism by which the Grouping Transform logic can hold back the output watermark until it is sure that it can satisfy the invariants described above.

Each Transform Executor has a \emph{watermark manager} which maintains a pair holds (the data hold and end-of-window/garbage-collection hold) per-window-per-key.
These are maintained separately so that they can be cleared and manipulated individually, but the overall watermark hold is simply their overall minimum.
Two holds are maintained for each window because some of the logic below requires us to know the data-derived hold even if the hold in effect is derived from the window only.

The data hold is, in effect, a reduction over all the element timestamps seen so far.
The exact reducing function is determined by the OutputTimeFn in use (specified by the user).
While the default simply sets all element timestamps to the end of their window---therefore making a reduction redundant as all inputs are always the same---other strategies are available such as `use the latest timestamp seen so far'.

The EOW/GC hold is a fallback used when we try to set a hold which cannot be honoured because either the output watermark has already progressed past it, or the input watermark has progressed beyond the EOW and therefore a timer to clear this hold won't be fired.
If we try to set an element hold in either of those situations, we automatically instead try to set the EOW/GC hold instead.

\newpage
An end-of-window hold is set in two situations:
\begin{itemize}
	\item The element is too late with respect to the output watermark to hold it back, but it may still be possible to include the element in an \verb|ON_TIME| pane.
	The EOW hold is placed to ensure that pane will not be considered late downstream.
	\item We must ensure that an \verb|ON_TIME| pane will be emitted for all windows which saw at least one element, even if that \verb|ON_TIME| pane is empty.
	Therefore, we place the EOW hold to ensure that this (possibly empty) pane will not be considered late downstream.
\end{itemize}

If the input watermark has progressed beyond the end-of-window, we can no longer place the EOW hold since a timer will not fire to clear it.
Instead, if the allowed lateness is non-zero, we try to set an additional garbage collection hold, which is analogous to the end-of-window hold but ensures that the panes we emit are at least non-droppable in the case they must be late.

When the trigger for a window fires, the hold for that window is cleared.
However, a garbage-collection hold may still remain for reasons described above.

When we talk about `placing a hold' below, we actually \emph{attempt} to place an element hold with a particular timestamp, at which point the above checks are performed and the appropriate hold set (or, possibly, no hold is set).
This allows for a decoupling of concerns in the two pieces of logic.

\section{Default trigger implementation}\label{apx:additional:example-default-trigger}

\Cref{lst:apx:additional:default-trigger} illustrates the decoupled manner in which triggers are specified as FSMs.

\begin{codelisting}
	\caption[An implementation of the default trigger as a FSM in Elixir.]{An implementation of the default trigger in Elixir. The trigger fires once at the end of the window, and fires again (indefinitely) on seeing any late data after that. Cf.\ the definition of the \exs{TriggerDriver} behaviour in \cref{lst:impl:trigger_driver}.}
	\label{lst:apx:additional:default-trigger}
	\begin{minted}{elixir}
defmodule Dataflow.DirectRunner.TriggerDrivers.Default do
  @behaviour Dataflow.DirectRunner.ReducingEvaluator.TriggerDriver

  alias Dataflow.{Trigger, Window, Utils.Time}
  require Time

  alias Dataflow.DirectRunner.TimingManager, as: TM

  defstruct [:window, :timing_manager]

  def init(%Trigger.Default{}, window, tm) do
    %__MODULE__{
      window: window,
      timing_manager: tm
    }
  end

  def process_element(%__MODULE__{window: window, timing_manager: tm} = state, _timestamp) do
    # If the end of the window has already been reached
    # then we are already ready to fire
    # and do not need to set a wake-up timer.
    event_time = TM.get_liwm tm
    if eow_reached?(window, event_time) do
      state
    else
      TM.set_timer(tm, {window, :default_trigger}, Window.max_timestamp(window), :event_time)
      state
    end
  end

  def merge(states, %__MODULE__{window: window, timing_manager: tm} = state) do
    # If the end of the window has already been reached
    # then we are already ready to fire
    # and do not need to set a wake-up timer.

    # clear any existing timers
    Enum.each states, fn %{window: window, timing_manager: tm} -> TM.clear_timers(tm, {window, :default_trigger}, :event_time) end

    event_time = TM.get_liwm tm

    if eow_reached?(window, event_time) do
      state
    else
      TM.set_timer(state.tm, {window, :default_trigger}, Window.max_timestamp(window), :event_time)
      state
    end
  end

  def should_fire?(state) do
    event_time = TM.get_liwm state.tm
    eow_reached?(state.window, event_time)
  end

  def fired(state) do
    state
  end

  def finished?(state) do
    false
  end

  defp eow_reached?(window, time) do
    time != :none && Time.after?(time, Window.max_timestamp(window))
  end
end
	\end{minted}

\end{codelisting}
