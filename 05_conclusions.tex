\chapter{Conclusions}\label{ch:concl}

The project achieved its stated goals and is a success.

The project has been successful in describing the semantics of the Beam Model to a level of detail necessary for a working implementation.

A working implementation of the Model was produced in the Elixir programming language, showing its suitability for developing systems such as this one.
The developer-friendliness of the resultant system was also shown, with paradigms and conventions of the Elixir ecosystem being a great match for the concepts of the Model.

The implementation was evaluated, and though it was built with the goal of academic implementation of the Model rather than speed, it displayed excellent performance characteristics when pitted against both the local Beam runner and the optimised Apache Beam runner.

Its suitability for real-world scenarios was shown by its performance while processing a large stream of tweets to produce hashtag autocomplete suggestions.

The implementation lacks some features such as the ability to support branching pipelines.
These would be relatively simple to add but have been omitted due to time constraints.
Further, the implementation forgoes many optimisations which could be made.
Other implementations have introduced Transform coalescing, where several Elementwise Transforms in a row could be merged into one which applies the compound operation.
There is also no automatic Transform-level parallelisation, where a Collection is automatically partitioned and processed concurrently/in parallel.
The lack of this optimisation proved to be the limiting factor when evaluating the Twitter Pipeline throughput.

The seamless distribution of the system across a cluster of nodes would be an important step towards handling production loads.
OTP and the BEAM VM provide fantastic tools and primitives which make the ecosystem an excellent fit for this kind of work, but distributed computation remains intrinsically difficult to control.

Finally, work is ongoing in the Beam community \cite{JIRA-retractions} to develop the semantics needed to support refinements and retractions as described in the original paper.
This is a fundamentally difficult problem to solve without an explosion in the size of cached state.
A general solution to this problem is likely to be an interesting area of research in the future.