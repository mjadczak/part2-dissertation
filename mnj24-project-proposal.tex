
\documentclass[11pt]{scrartcl}
\usepackage{geometry} % see geometry.pdf on how to lay out the page. There's lots.
\usepackage[usenames,dvipsnames,svgnames,table]{xcolor}
\usepackage{graphicx}
\usepackage{caption}
\usepackage{setspace}
\usepackage{enumerate}
\usepackage{adjustbox}
\usepackage{tabu}
\usepackage[T1]{fontenc}
\usepackage{amsmath}
\usepackage{ amssymb }
\usepackage{cutwin}
%\usepackage{bm}
\usepackage{url}
\usepackage{centernot}
\usepackage{mathtools}
\usepackage{cancel}
\usepackage{turnstile}
\usepackage{pdflscape}
\usepackage{tikz}
\usepackage{siunitx}
\usepackage[full]{textcomp}
\usepackage[osf]{newpxtext} % osf for text, not math
\usepackage{cabin} % sans serif
\usepackage[varqu,varl]{inconsolata} % sans serif typewriter
\usepackage[bigdelims,vvarbb]{newpxmath} % bb from STIX
\usepackage[cal=boondoxo]{mathalfa} % mathcal



\setlength{\oddsidemargin}{-0.4mm}    % 25 mm left margin - 1 in
\setlength{\evensidemargin}{\oddsidemargin}
\setlength{\topmargin}{-5.4mm}        % 20 mm top margin - 1 in
\setlength{\textwidth}{160mm}         % 20/25 mm right margin
\setlength{\textheight}{237mm}        % 20 mm bottom margin
\setlength{\headheight}{5mm}
\setlength{\headsep}{5mm}
\setlength{\parindent}{0mm}
\setlength{\parskip}{\medskipamount}
\renewcommand\baselinestretch{1.2} % thesis format (not needed for techreport)
% don't let large figures hijack entire pages
\renewcommand\topfraction{.9}
\renewcommand\textfraction{.1}
\renewcommand\floatpagefraction{.8}

\newcommand{\mlpos}{}

\makeatletter
\DeclareFontFamily{OMX}{MnSymbolE}{}
\DeclareSymbolFont{MnLargeSymbols}{OMX}{MnSymbolE}{m}{n}
\SetSymbolFont{MnLargeSymbols}{bold}{OMX}{MnSymbolE}{b}{n}
\DeclareFontShape{OMX}{MnSymbolE}{m}{n}{
    <-6>  MnSymbolE5
   <6-7>  MnSymbolE6
   <7-8>  MnSymbolE7
   <8-9>  MnSymbolE8
   <9-10> MnSymbolE9
  <10-12> MnSymbolE10
  <12->   MnSymbolE12
}{}
\DeclareFontShape{OMX}{MnSymbolE}{b}{n}{
    <-6>  MnSymbolE-Bold5
   <6-7>  MnSymbolE-Bold6
   <7-8>  MnSymbolE-Bold7
   <8-9>  MnSymbolE-Bold8
   <9-10> MnSymbolE-Bold9
  <10-12> MnSymbolE-Bold10
  <12->   MnSymbolE-Bold12
}{}

\let\llangle\@undefined
\let\rrangle\@undefined
\DeclareMathDelimiter{\llangle}{\mathopen}%
                     {MnLargeSymbols}{'164}{MnLargeSymbols}{'164}
\DeclareMathDelimiter{\rrangle}{\mathclose}%
                     {MnLargeSymbols}{'171}{MnLargeSymbols}{'171}
\makeatother


% See the ``Article customise'' template for come common customisations


%%% BEGIN DOCUMENT
\begin{document}

\begin{center}
\Large
Computer Science Tripos -- Part II -- Project Proposal\\[4mm]
\LARGE
An implementation of Google Dataflow in the Elixir programming language\\[4mm]

\large
M.~N.~Jadczak \texttt{<mnj24>}, Robinson College

Originator: Dr M.~Kleppmann

14 October 2016
\end{center}

\vspace{5mm}

\textbf{Project Supervisor:} Dr A.~Beresford

\textbf{Director of Studies:} Dr A.~Beresford

\textbf{Project Overseers:} Dr S.~Holden  \& Dr S.~Teufel

% Main document

\section*{Introduction}

\emph{The problem to be addressed.}

\section*{Starting point}

\emph{Describe existing state of the art, previous work in this area,
  libraries and databases to be used. Describe the state of any
  existing codebase that is to be built on.}

\section*{Resources required}

\emph{A note of the resources required and confirmation of access.}

\section*{Work to be done}

\emph{Describe the technical work.}

\section*{Success citeria}

\emph{Describe what you expect to be able to demonstrate at the
end of the project and how you are going to evaluate your achievement.}


\section*{Possible extensions}

{\em Potential further envisaged evaluation metrics or extensions.}


\section*{Timetable}

\emph{A workplan of perhaps ten or so two-week work-packages,
as well as milestones to be achieved along the way. Provide a
target date for each milestone.}

The planned starting date is 17/10/2016. My plan assumes the division of the work into ten work-packages, detailed below:

\begin{enumerate}

\item \textbf{Michaelmas weeks 2--4:}

\item \textbf{Michaelmas weeks 5--6:}

\item \textbf{Michaelmas weeks 7--8:}

\item \textbf{Michaelmas vacation:}

\item \textbf{Lent weeks 0--2:}

\item \textbf{Lent weeks 3--5:} 

\item \textbf{Lent weeks 6--8:}

\item \textbf{Easter vacation:} 

\item \textbf{Easter weeks 0--2:}

\item \textbf{Easter week 3:} 

\end{enumerate}


\end{document}